%%%%%%%%%%%%%%%%%%%%%%%%%%%%%%%%%%%%%%%%%%%%%%%%%%%%%%%%%%%%%%%%%%%%%%%%%%%
% FILE   : ChatterDoc.tex
% AUTHOR : (C) Copyright 2013 by Peter C. Chapin
% SUBJECT: Chatter
%
% This document describes the Chatter instant messaging system. This system is based on
% middleware and intended to serve as a vehicle for learning and exploring middleware
% technologies. It is also intended to be useful for its own sake as well.
%
%       Peter C. Chapin
%       Computer Information Systems
%       Vermont Technical College
%       Williston, VT. 05495
%       PChapin@vtc.vsc.edu
%%%%%%%%%%%%%%%%%%%%%%%%%%%%%%%%%%%%%%%%%%%%%%%%%%%%%%%%%%%%%%%%%%%%%%%%%%%

% ++++++++++++++++++++++++++++++++
% Preamble and global declarations
% ++++++++++++++++++++++++++++++++
\documentclass[twocolumn]{article}
%\pagestyle{headings}
%\setlength{\parindent}{0em}
%\setlength{\parskip}{1.75ex plus0.5ex minus0.5ex}

% Eventually a separate macro file might be desirable.
\newcommand{\Chatter}{\texttt{Chatter}}

% +++++++++++++++++++
% The document itself
% +++++++++++++++++++
\begin{document}

% ----------------------
% Title page information
% ----------------------
\title{Chatter Documentation}
\author{\copyright\ Copyright 2013 by Peter C. Chapin}
\date{Last Revised: April 8, 2013}
\maketitle

\section*{Introduction}

\Chatter\ is a simple but flexible instant messaging system based on middleware technology.
Currently \Chatter\ is most mature in C++ on the Ice platform\footnote{\tt
  http://www.zeroc.com/}. There also exists a rudimentary C++ CORBA implementation and a
rudimentary Java Ice implementation. It is my hope that eventually \Chatter\ will be implemented
in a variety of programming languages on a variety of middleware systems. In fact it is my
primary intention for \Chatter\ to be an educational vehicle for exploring middleware.

The \Chatter\ protocol is defined by a small collection of interfaces. Objects implementing
these interfaces can interact in many potentially useful and interesting ways. Because \Chatter\
is quite general, many applications besides instant messaging are possible. For example, objects
that monitor a device could send messages to objects that maintain a log file. Chat room objects
could accept messages and forward them to a collection of other objects. Special filter objects
could be created that remove or modify some of the messages that pass through them. Such filters
could potentially offer services such as compression, encryption, or character set conversions
to name a few possibilities.

Currently \Chatter\ is implemented as a library of classes supporting some of the options
mentioned above. Because the system is so flexible there is no single \Chatter\ application.
Instead several sample applications are also provided in this distribution.

The code base is divided by middleware technology since the selection of a middleware platform
tends to permeate the rest of the code. The Ice version of \Chatter\ supports C++ and Java so
different folders are used for the two languages. I hope that in the future additional
programming languages will be supported by \Chatter. It may be possible to arrange the system so
that, for each programming language, a single code base supports multiple middleware platforms.
However, such an organization might be difficult to create and maintain.

In addition to providing a medium for experimenting with middleware platforms, I also hope that
\Chatter\ will provide a good test bed for experimenting with distributed authorization systems.
This is a reflection of my interest in distributed authorization in general. I imagine a
\Chatter\ system where each user creates a policy specifying who can access the methods of their
objects. With each method invocation appropriate credentials would be passed from client to
server along with the method's arguments. A policy compliance checker on the server side of the
connection would then verify that the client's credentials are sufficient to allow access.

I feel that an instant messaging system would be a good test bed for practical distributed
authorization solutions. Users may wish to control who can send them messages and the
administrators of resources like loggers, chat room servers, and other entities may wish to
control access to those resources in complex ways. Thus \Chatter\ should provide a rich and
interesting environment for distributed authorization experiments.

\section{Design}

The file \texttt{Chatter.ice} in the top level directory of the \Chatter\ source tree contains
the interface definitions of the various \Chatter\ objects. Consult that file for specific
documentation on those interfaces.

\textit{A more extensive discussion of the design will hopefully appear in a later version of
  this document.}

\section{Discussion}

One interesting issue is related to cycles in the \Chatter\ network. One object might forward
messages to a second object where they are in turn forwarded, directly or indirectly, back to
the first. In this case messages would circulate around such a cycle infinitely unless special
steps are taken to prevent them from doing so.

% Should include a figure of the email message splitter described below.

Cycles are not necessarily bad. In fact, they allow a basic looping structure in \Chatter\
networks that can be used to perform useful computation. Imagine a Splitter object that accepts
RFC-5822 formatted email messages and splits off the header of such messages. It might send the
header to one destination and the body to another. Suppose that the object receiving the body
checks for a header and, if it finds one (because the original message was a wrapper), sends the
body back to the splitter. This network will cause a deeply nested message to loop, with its
headers being split of repeatedly until the final, innermost body has been reached.

Thus even if the \Chatter\ objects do not use loops or recursion, computations involving loops
or recursion can be executed by a suitably configured network of such objects. Note that unlike
an ordinary program (or an ordinary Turing Machine), a \Chatter\ network is subject to dynamic
reconfiguration. As its ``program'' executes new control structures could come into being and
old control structures could be dismantled. It is interesting to speculate on what effect this
ability has on the computational abilities of a \Chatter\ network. Can this ability be used to
design novel and useful algorithms for solving certain problems?

\subsection{Removing Cycles}

In an instant messaging application, cycles are undesirable. One way to remove them would be to
stamp each message with a unique ID number when that message is freshly generated (that is, not
in response to an incoming message). Messages passed between \Chatter\ nodes would then be a
structure containing the text of the message and the ID number. When a message is filtered, the
output message gets the same ID number as the corresponding incoming message. Objects that
accept messages would have to keep track of what ID numbers they have seen and ignore messages
that are duplicates.

There are several problems with this scheme:

\begin{enumerate}

\item How are the ID numbers generated? They must be globally unique yet objects that introduce
  fresh messages onto the network are not aware of each other and can't synchronize.

\item How many IDs should an object track? Should the number be a count of different IDs seen so
  far, or based on a certain time interval? If an object remembers too few IDs, cycles with a
  long cycle time won't be broken. However, it is a resource burden to remember a large number
  of IDs.

\item If an incoming message produces several outgoing messages, should any or all of the
  outgoing messages be considered fresh? How does one decide?

\end{enumerate}

Another approach would be to give every \emph{object} a globally unique ID and then attach to
each message a list of objects through which the message has passed. When a message is received,
the receiving object scans the list looking for its ID and ignores the message if it is found.

This approach means that the objects don't have to worry about managing a database of messages
they have seen and don't have to apply an arbitrary limit to the size of that database. Thus
objects are more lightweight than they would be in the earlier system.

Several problems remain:

\begin{enumerate}

\item We still need to generate globally unique IDs (although probably not as many since in most
  cases there should be fewer objects than messages).

\item Message overhead is greater. This might be a serious problem if messages typically go
  through many objects.

\item Questions remain regarding the handling of multiple outgoing messages. Do all of the
  outgoing messages get the entire ID list attached to the corresponding incoming message?

\end{enumerate}

The second approach for cycle-breaking would be best in an environment where the network is
quite fast and the nodes have limited resources. The first approach would be better in exactly
the opposite situation.

\section{Building \Chatter}

This section describes how to build and install \Chatter\ starting with the source distribution.

\subsection{Ice with C++}

If you are interested in building \Chatter\ for Ice you must first have the Ice headers,
libraries, and support applications installed. Both Linux and Windows are supported build
platforms.

Download Ice from ZeroC and install it according to the instructions contained in the Ice
distribution. Before building \Chatter\ you will need to define an \texttt{ICE\_HOME}
environment variable that points at the top level folder or directory where the Ice system is
installed. The \Chatter\ Makefiles use this environment variable to locate the Ice headers and
libraries. In addition you should add the Ice binary folder or directory to your \texttt{PATH}
environment variable. The \Chatter\ Makefiles need access to Ice tools such as
\texttt{slice2cpp} during the build.

\subsection{Ice with Java}

The Java version of \Chatter\ is build with IntelliJ IDEA. After downloading and installing Ice
according to the instructions contained in the Ice distribution you must then use
\texttt{slice2java} to create the necessary stubs and skeletons. Execute \texttt{slice2java} in
the \texttt{src} folder under \texttt{Ice/Java}. You will need to provide a relative path to the
common \texttt{Chatter.ice} file that is shared with the C++ code base.

After building the stubs and skeletons you can then load the IntelliJ project from
\texttt{Ice/Java} and build the system in the usual way.

\section{\Chatter\ for Developers}

\textit{This section is intended for developers who wish to extend \Chatter\ interfaces to
  create new objects that can participate in a \Chatter\ network or who wish to enhance the
  existing Chatter system. It provides information about how to program \Chatter.}

\subsection{Ice}

At the moment the \Chatter\ interfaces are only documented in \texttt{Chatter.ice} and any slice
specifications included by \texttt{Chatter.ice}. Please read the comments in the slice files for
information about each interface.

Ideas for enhancements to the existing \Chatter\ system are included in comments in the various
source files. Please refer to the source code for more information.

\section{\Chatter\ for Users}

\textit{This section is intended for users of the system (non-programmers). It provides
  information about how to set up and run the demonstration applications.}

When you run your application you'll need to specify the address of the naming service or you
will get exceptions when your programs attempt to resolve the initial references. Exactly how
this is done is vendor dependent.

\end{document}

